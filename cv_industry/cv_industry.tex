\documentclass{article}
\usepackage[utf8]{inputenc}
\usepackage[margin=1.5cm]{geometry}
\usepackage{natbib}
\usepackage{amssymb, amsmath}
\usepackage{xcolor}
\usepackage[colorlinks=True,urlcolor=blue]{hyperref}
\usepackage{array} % for defining new column types
\usepackage{fontawesome} % for icons
\usepackage{comment} % for commenting out sections
\usepackage{titlesec} % for custom section titles

% Define new column types for the tables
\newcolumntype{L}{>{\raggedleft}p{0.175\textwidth}}
\newcolumntype{R}{p{0.765\textwidth}}

% Custom section titles
\titleformat{\section}{\Large\bfseries}{\thesection}{}{}[{\titlerule[0.5pt]}]
\titleformat{\subsection}{\normalsize\bfseries}{\thesection}{}{}

\begin{document}

\small % Reduce font size for the whole document

\begin{center}
\huge
\textbf{Jordan Zambrano} \\
\normalsize
Quantitative Researcher | Python Developer \\
\end{center}

\begin{minipage}[ht]{0.6\linewidth}
	\faEnvelope~~\textbf{Email}: jordan.zambrano@yachaytech.edu.ec/jordanzambrano12@gmail.com \\
	\faGithub~~\textbf{GitHub}: \href{https://github.com/yourprofile}{github.com/yourprofile} \\
	\faLinkedin~~\textbf{LinkedIn}: \href{https://www.linkedin.com/in/jordan-zambrano}{linkedin.com/in/jordan-zambrano} \\
\end{minipage}
\begin{minipage}[ht]{0.4\linewidth}
	\faMapMarker~~\textbf{Location}: Ecuador \\
\end{minipage}

\section*{Summary}
Quantitative researcher and Python developer with expertise in algorithmic trading, data science, and numerical modeling. Experienced in developing trading algorithms, backtesting strategies, and working with large-scale financial datasets. Passionate about applying computational techniques to financial markets.

\section*{Skills}
\begin{itemize}
    \item Programming: Python (NumPy, Pandas, Scikit-learn), SQL, C++
    \item Financial Modeling: Backtesting, Risk Management, Statistical Arbitrage
    \item Data Science: Machine Learning, Time Series Analysis, Bayesian Methods
    \item Tools: Jupyter, Git, Linux, Docker, Cloud Computing (AWS, GCP)
\end{itemize}

\section*{Experience}
\textbf{Independent Quantitative Researcher} \hfill 2023 - Present \\
\textit{Developing algorithmic trading strategies, market analysis tools, and backtesting frameworks.}
\begin{itemize}
    \item Designed and implemented high-frequency trading (HFT) models using Binance API.
    \item Conducted quantitative research on statistical arbitrage and volatility modeling.
    \item Developed backtesting frameworks to validate and optimize trading strategies.
\end{itemize}

\textbf{Python Developer – Financial Analytics Project} \hfill 2022 - 2023 \\
\textit{Designed data pipelines and visualization dashboards for financial market analysis.}
\begin{itemize}
    \item Built automated scripts for data collection and processing from financial APIs.
    \item Implemented machine learning models for market trend predictions.
    \item Developed interactive dashboards using Plotly and Streamlit.
\end{itemize}

\section*{Education}
\textbf{B.Sc. in Physics} \hfill Yachay Tech University, Ecuador \\
\textit{Specialization in Computational and Theoretical Physics.}

\section*{Projects}
\textbf{Cryptocurrency Market Scanner} \hfill 2023 \\
Developed a Python-based scanner to identify trading opportunities in real-time using Binance API.
\begin{itemize}
    \item Implemented statistical filters to detect arbitrage and breakout patterns.
    \item Optimized code for low-latency execution in high-frequency environments.
\end{itemize}

\textbf{Portfolio Optimization Tool} \hfill 2022 \\
Created a Python tool using Modern Portfolio Theory (MPT) to optimize asset allocation.
\begin{itemize}
    \item Integrated Monte Carlo simulations to assess risk and returns.
    \item Developed visualization tools to analyze portfolio performance over time.
\end{itemize}

\section*{Certifications}
\begin{itemize}
    \item Algorithmic Trading Specialization – Coursera
    \item Python for Finance – Udemy
    \item Machine Learning in Finance – edX
\end{itemize}

\section*{Languages}
\begin{itemize}
    \item Spanish: Native
    \item English: B2/C1
\end{itemize}

\end{document}
